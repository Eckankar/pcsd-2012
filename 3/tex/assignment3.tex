\documentclass[11pt,a4paper]{article}

\usepackage[utf8]{inputenc}
\usepackage[english]{babel}
\usepackage{graphicx}
\usepackage{algorithm}
\usepackage{algpseudocode}

\title{Principles of Computer System Design\\Assignment 3}
\author{Jacob Wejendorp}

\begin{document}
\maketitle
\section{Theory}
\subsection{Question 1, Communication abstractions}
  \subsubsection{Synchronous communication with async primitives}

\begin{algorithm}[H]
\begin{algorithmic}
  \State $handle \leftarrow bufferSend(data)$
    \Comment{Add to send buffer and get wait handle}
  \State$t \leftarrow$ {\tt currentTime}
 \While{! handle.isReceived()}
    \If{$t -  {\tt currentTime} > lim$}
      \Return Timeout
    \EndIf
  \State Sleep()
 \EndWhile

\State \Return{Ack}

\end{algorithmic}
	\caption{SyncSend by BusyWait for Ack}
\end{algorithm}

\begin{algorithm}[H]
\begin{algorithmic}
 \State$t \leftarrow$ {\tt currentTime}
 \While{bufferRecv() is empty}
    \Comment{Check recv buffer}
    \If{$t -  {\tt currentTime} > lim$}
      \Return Timeout
    \Else
      \State Sleep()
    \EndIf

 \EndWhile

\State \Return{Data}

\end{algorithmic}
	\caption{SyncRecv by BusyWait for Data}
\end{algorithm}


\subsubsection{Asynchronous communication via sync primitives}
If threads are allowed, the simple answer is to defer the work to a worker thread via a buffer.
The transient part is "ensured" by allowing the requests to timeout without retries.

\subsubsection{Persistent async comms via RPCs}

The api would be a message queueing system, allowing the processes to "check their mailboxes" at different times.
This could be done by using Message-Oriented Middleware, giiving the APi the methods
\begin{itemize}
\item Subscribe
\item Publish
\end{itemize}
The MOM instance would then work as a buffer for two communicating processes.
Subscription could be done via polling or via an open socket on the client.


\subsection{Question 3: Reliability}

\begin{enumerate}
  \item $(1-p)^2$ where $p$ is the probability that a single link fails.
  \item This would require two links to fail, and thus the chance is $1-p^2$.
  \item The difference in reliability is then $|(1-0.000001)^2 - (1-0.0001^2)| = 1.99 \times 10^-6$ in favor of the fully connected network.
\end{enumerate}


\section{Progamming Task}

\subsection{Question 1: KeyValueBaseMaster}
% Describe what you had to change
In our original implementation everything was collected in a single file. The new interfaces encouraged a clear division of responsibilites.
We have chosen to build on the simpler version of the service from assignment 1.
The master is dependent on the Write operations defined in {\tt IKeyValueBaseMaster}, so a dedicated Master implementation is made by
moving write operations into {\tt KeyValueBaseMasterImpl} and read operations into {\tt KeyValueBaseReplicaImpl}.
The master connects to the Slaves after configuration by configuring the replicator. In short, partitioning of functionalities and adding the replicator to the master.

\subsection{Question 2: KeyValueBaseSlave}
The {\tt KeyValueBaseSlave} depends on the {\tt KeyValueBaseReplicaImpl}, and the {\tt IKeyValueBaseSlave} interface. The slave is easily implemented by
implementing {\tt LogApply} in this class. This is done by reading out the values from the {\tt LogRecord} and applying the changes directly to the inherited {\tt IndexImpl},
and updating the {\tt LastLSN} of the slave.
The {\tt LastLSN} field also needs to be maintained in the master, which means adding some extra logic to the write methods from Question 1.

\subsection{Question 3: KeyValueBaseProxy}
The proxies enforce timestamp order by keeping a local {\tt LastLSN} field, and retrying requests on a different client when an older timestamp is received.
Load balancing is done simply by randomization amongst the slaves, while only querying the master on writes or when all other timestamp checks fail.
The proxies rely solely on Timeouts from {\tt KeyValueBaseMasterClient} and {\tt KeyValueBaseSlaveClient} to determine live nodes.




\end{document}

